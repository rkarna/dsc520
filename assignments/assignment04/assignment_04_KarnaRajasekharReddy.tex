% Options for packages loaded elsewhere
\PassOptionsToPackage{unicode}{hyperref}
\PassOptionsToPackage{hyphens}{url}
%
\documentclass[
]{article}
\usepackage{lmodern}
\usepackage{amssymb,amsmath}
\usepackage{ifxetex,ifluatex}
\ifnum 0\ifxetex 1\fi\ifluatex 1\fi=0 % if pdftex
  \usepackage[T1]{fontenc}
  \usepackage[utf8]{inputenc}
  \usepackage{textcomp} % provide euro and other symbols
\else % if luatex or xetex
  \usepackage{unicode-math}
  \defaultfontfeatures{Scale=MatchLowercase}
  \defaultfontfeatures[\rmfamily]{Ligatures=TeX,Scale=1}
\fi
% Use upquote if available, for straight quotes in verbatim environments
\IfFileExists{upquote.sty}{\usepackage{upquote}}{}
\IfFileExists{microtype.sty}{% use microtype if available
  \usepackage[]{microtype}
  \UseMicrotypeSet[protrusion]{basicmath} % disable protrusion for tt fonts
}{}
\makeatletter
\@ifundefined{KOMAClassName}{% if non-KOMA class
  \IfFileExists{parskip.sty}{%
    \usepackage{parskip}
  }{% else
    \setlength{\parindent}{0pt}
    \setlength{\parskip}{6pt plus 2pt minus 1pt}}
}{% if KOMA class
  \KOMAoptions{parskip=half}}
\makeatother
\usepackage{xcolor}
\IfFileExists{xurl.sty}{\usepackage{xurl}}{} % add URL line breaks if available
\IfFileExists{bookmark.sty}{\usepackage{bookmark}}{\usepackage{hyperref}}
\hypersetup{
  pdftitle={ASSIGNMENT 4},
  pdfauthor={Rajasekhar Reddy Karna},
  hidelinks,
  pdfcreator={LaTeX via pandoc}}
\urlstyle{same} % disable monospaced font for URLs
\usepackage[margin=1in]{geometry}
\usepackage{longtable,booktabs}
% Correct order of tables after \paragraph or \subparagraph
\usepackage{etoolbox}
\makeatletter
\patchcmd\longtable{\par}{\if@noskipsec\mbox{}\fi\par}{}{}
\makeatother
% Allow footnotes in longtable head/foot
\IfFileExists{footnotehyper.sty}{\usepackage{footnotehyper}}{\usepackage{footnote}}
\makesavenoteenv{longtable}
\usepackage{graphicx,grffile}
\makeatletter
\def\maxwidth{\ifdim\Gin@nat@width>\linewidth\linewidth\else\Gin@nat@width\fi}
\def\maxheight{\ifdim\Gin@nat@height>\textheight\textheight\else\Gin@nat@height\fi}
\makeatother
% Scale images if necessary, so that they will not overflow the page
% margins by default, and it is still possible to overwrite the defaults
% using explicit options in \includegraphics[width, height, ...]{}
\setkeys{Gin}{width=\maxwidth,height=\maxheight,keepaspectratio}
% Set default figure placement to htbp
\makeatletter
\def\fps@figure{htbp}
\makeatother
\setlength{\emergencystretch}{3em} % prevent overfull lines
\providecommand{\tightlist}{%
  \setlength{\itemsep}{0pt}\setlength{\parskip}{0pt}}
\setcounter{secnumdepth}{-\maxdimen} % remove section numbering

\title{ASSIGNMENT 4}
\author{Rajasekhar Reddy Karna}
\date{2020-09-23}

\begin{document}
\maketitle

\hypertarget{markdown-basics}{%
\subsection{Markdown Basics}\label{markdown-basics}}

R provides authoring framework for Data Science. Users can use Markdowns
to connect to data and run code. Also generates high quality reports to
share with audience.

R Markdown is plain text file with extension as .Rmd. It has 3 types of
content: - Code chunks to run - Text to display - Metadata to guide
build process

Users can customize code on how text to display or parameterize to use
in render time.

R Markdown support dozens of static and dynamic output formats, such as
HTML, pdf, word, slide shows, notebook, latex, etc\ldots Easy to track
in version control tools like GIT and easy to deploy.

\hypertarget{favorite-foods}{%
\subsection{Favorite Foods}\label{favorite-foods}}

\begin{itemize}
\tightlist
\item
  Biryani
\item
  American / Italian food
\item
  Thai food
\end{itemize}

\hypertarget{images}{%
\subsection{Images}\label{images}}

\begin{figure}
\centering
\includegraphics{C:/Users/vahin/Documents/GitHub/dsc520/completed/assignment04/plots/10-all-cases-log.png}
\caption{All Cases (Log Plot)}
\end{figure}

\hypertarget{add-a-quote}{%
\subsection{Add a Quote}\label{add-a-quote}}

Life is full of emotions. Give a touch of positive hope to add smile to
those emotions!

\hypertarget{add-an-equation}{%
\subsection{Add an Equation}\label{add-an-equation}}

\(\begin{equation}  \frac{1}{\sqrt x} \end{equation}\)

\hypertarget{add-a-footnote}{%
\subsection{Add a Footnote}\label{add-a-footnote}}

Summary always makes easy to understand.

\hypertarget{add-citations}{%
\subsection{Add Citations}\label{add-citations}}

\begin{itemize}
\tightlist
\item
  R for Everyone. Pearson Education, 2017. 2nd Edition.
\item
  Discovering Statistics Using R. Sage Publications, 2012.
\end{itemize}

\hypertarget{inline-code}{%
\subsection{Inline Code}\label{inline-code}}

\hypertarget{ny-times-covid-19-data}{%
\subsection{NY Times COVID-19 Data}\label{ny-times-covid-19-data}}

\includegraphics{assignment_04_KarnaRajasekharReddy_files/figure-latex/unnamed-chunk-2-1.pdf}

\hypertarget{r4ds-height-vs-earnings}{%
\subsection{R4DS Height vs Earnings}\label{r4ds-height-vs-earnings}}

\includegraphics{assignment_04_KarnaRajasekharReddy_files/figure-latex/unnamed-chunk-3-1.pdf}

\hypertarget{tables}{%
\section{Tables}\label{tables}}

\hypertarget{knitr-table-with-kable}{%
\subsection{Knitr Table with Kable}\label{knitr-table-with-kable}}

\begin{longtable}[]{@{}llllr@{}}
\caption{One Ring to Rule Them All}\tabularnewline
\toprule
name & race & in\_fellowship & ring\_bearer & age\tabularnewline
\midrule
\endfirsthead
\toprule
name & race & in\_fellowship & ring\_bearer & age\tabularnewline
\midrule
\endhead
Aragon & Men & TRUE & FALSE & 88\tabularnewline
Bilbo & Hobbit & FALSE & TRUE & 129\tabularnewline
Frodo & Hobbit & TRUE & TRUE & 51\tabularnewline
Galadriel & Elf & FALSE & FALSE & 7000\tabularnewline
Sam & Hobbit & TRUE & TRUE & 36\tabularnewline
Gandalf & Maia & TRUE & TRUE & 2019\tabularnewline
Legolas & Elf & TRUE & FALSE & 2931\tabularnewline
Sauron & Maia & FALSE & TRUE & 7052\tabularnewline
Gollum & Hobbit & FALSE & TRUE & 589\tabularnewline
\bottomrule
\end{longtable}

\hypertarget{pandoc-table}{%
\subsection{Pandoc Table}\label{pandoc-table}}

setwd(``C:/Users/vahin/Documents/GitHub/dsc520/assignments/assignment04/'')
Name Race In Fellowship? Is Ring Bearer? Age ------- ---------
------------------ -------------------- ------- Aragon Men Yes No 88
Bilbo Hobbit No Yes 129 Frodo Hobbit Yes Yes 51 Sam Hobbit Yes Yes 36
Sauron Maia No Yes 7052 ------- --------- ------------------
-------------------- -------

\hypertarget{references}{%
\section{References}\label{references}}

\url{https://rmarkdown.rstudio.com/lesson-1.html} --- R Markdown basics
R for Everyone. Pearson Education, 2017. 2nd Edition.
\url{https://rstudio.com/wp-content/uploads/2016/03/rmarkdown-cheatsheet-2.0.pdf?_ga=2.247973831.1388722509.1600630414-1304384236.1598891840}
-- R Markdown cheat sheet for reference. Slake for students feedback and
reference comments.

\end{document}
